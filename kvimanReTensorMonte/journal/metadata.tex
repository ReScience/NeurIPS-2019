% DO NOT EDIT - automatically generated from metadata.yaml

\def \codeURL{https://github.com/LinuNils/TMC\_reproduced}
\def \codeDOI{10.5281}
\def \dataURL{}
\def \dataDOI{}
\def \editorNAME{Koustuv Sinha}
\def \editorORCID{0000-0002-2803-9236}
\def \reviewerINAME{}
\def \reviewerIORCID{}
\def \reviewerIINAME{}
\def \reviewerIIORCID{}
\def \dateRECEIVED{15 February 2020}
\def \dateACCEPTED{}
\def \datePUBLISHED{}
\def \articleTITLE{[Re] Tensor Monte Carlo: Particle Methods for the GPU Era}
\def \articleTYPE{Replication}
\def \articleDOMAIN{NeurIPS 2019 Reproducibility Challenge}
\def \articleBIBLIOGRAPHY{bibliography.bib}
\def \articleYEAR{2020}
\def \reviewURL{https://openreview.net/forum?id=BJxUSaczTH}
\def \articleABSTRACT{In this work, we reproduce what we believe are the most important results presented in the Tensor Monte Carlo paper (Aitchison [2019]), where we also provide our reimplementation code. The original results in the TMC paper were attained via a PyTorch (Paszke et al. [2017]) implementation. In an attempt to ease understanding for those unfamiliar with PyTorch, we contribute with a TensorFlow (Abadi et al. [2015]) implementation. Additionally, as we found the TMC architecture non-trivial to understand, we aim to ease understanding for future users by complementing the textual description of the model with an algorithmic description and a depiction of the model.}
\def \replicationCITE{Laurence Aitchison. Tensor monte carlo: particle methods for the gpu era. In Advances in NeuralInformation Processing Systems, pages 7146–7155, 2019}
\def \replicationBIB{\cite{tmc}}
\def \replicationURL{http://papers.nips.cc/paper/8936-tensor-monte-carlo-particle-methods-for-the-gpu-era.pdf}
\def \replicationDOI{1806.08593}
\def \contactNAME{Oskar Kviman}
\def \contactEMAIL{okviman@kth.se}
\def \articleKEYWORDS{deep learning, variational autoencoders, rescience c, python, tensorflow}
\def \journalNAME{ReScience C}
\def \journalVOLUME{6}
\def \journalISSUE{1}
\def \articleNUMBER{}
\def \articleDOI{}
\def \authorsFULL{Oskar Kviman, Linus Nilsson and Martin Larsson}
\def \authorsABBRV{O. Kviman, L. Nilsson and M. Larsson}
\def \authorsSHORT{Kviman, Nilsson and Larsson}
\title{\articleTITLE}
\date{}
\author[1,\orcid{0000-0002-6369-712X}]{Oskar Kviman}
\author[1,\orcid{0000-0002-4232-1094}]{Linus Nilsson}
\author[1,\orcid{0000-0002-9151-5119}]{Martin Larsson}
\affil[1]{KTH Royal Institute of Technology, Stockholm, Sweden}
