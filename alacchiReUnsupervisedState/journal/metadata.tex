% DO NOT EDIT - automatically generated from metadata.yaml

\def \codeURL{https://github.com/rescience-c/template}
\def \codeDOI{}
\def \dataURL{}
\def \dataDOI{}
\def \editorNAME{Koustuv Sinha}
\def \editorORCID{0000-0002-2803-9236}
\def \reviewerINAME{}
\def \reviewerIORCID{}
\def \reviewerIINAME{}
\def \reviewerIIORCID{}
\def \dateRECEIVED{}
\def \dateACCEPTED{}
\def \datePUBLISHED{}
\def \articleTITLE{[Re] Unsupervised Representation Learning in Atari}
\def \articleTYPE{Replication}
\def \articleDOMAIN{NeurIPS 2019 Reproducibility Challenge}
\def \articleBIBLIOGRAPHY{bibliography.bib}
\def \articleYEAR{2020}
\def \reviewURL{}
\def \articleABSTRACT{In this study, we performed some ablations on the main model developed in the paper 	extbf{Unsupervised Representation Learning in Atari} as part of the 2019 NeurIPS Reproducibility Challenge. In this paper, Anand et. al introduce a new learning method called SpatioTemporal DeepInfoMax (STDIM), which is an unsupervised method that aims at learning state representations by maximizing particular forms of mutual information between a series of observations. Our work focuses on recreating a subset of their results, along with hyperparameter tuning, slightly altering the STDIM learning objective, and altering the receptive field of the encoder model that Anand et. al introduce in their article. We also suggest directions for further expanding the STDIM method. Our results also suggest that creating an ensemble model would allow for further boosting of the effectiveness of this model.}
\def \replicationCITE{A.Anand, E.Racah, S.Ozair, Y.Bengio, M.-A.Côté, and R.D.Hjelm. Unsupervised State Representation Learning in Atari. 2019.}
\def \replicationBIB{\cite{main_article}}
\def \replicationURL{https://arxiv.org/pdf/1906.08226.pdf}
\def \replicationDOI{}
\def \contactNAME{Gabriel Alacchi}
\def \contactEMAIL{gabriel.alacchi@mail.mcgill.ca}
\def \articleKEYWORDS{rescience c, rescience x}
\def \journalNAME{ReScience C}
\def \journalVOLUME{6}
\def \journalISSUE{1}
\def \articleNUMBER{}
\def \articleDOI{}
\def \authorsFULL{Gabriel Alacchi, Guillaume Lam and Carl Perreault-Lafleur}
\def \authorsABBRV{G. Alacchi, G. Lam and C. Perreault-Lafleur}
\def \authorsSHORT{Alacchi, Lam and Perreault-Lafleur}
\title{\articleTITLE}
\date{}
\author[1,$\dagger$]{Gabriel Alacchi}
\author[1,$\dagger$]{Guillaume Lam}
\author[1,$\dagger$]{Carl Perreault-Lafleur}
\affil[1]{School of Computer Science, McGill University, Montreal, Canada}
\affil[$\dagger$]{Equal contributions}
