% DO NOT EDIT - automatically generated from metadata.yaml

\def \codeURL{https://github.com/MacroMayhem/OnMixup}
\def \codeDOI{}
\def \codeSWH{swh:1:dir:824bff001a04ed1c4bb78dc6fbed52fb4470b7a5}
\def \dataURL{}
\def \dataDOI{}
\def \editorNAME{Koustuv Sinha}
\def \editorORCID{0000-0002-2803-9236}
\def \reviewerINAME{Anonymous Reviewers}
\def \reviewerIORCID{}
\def \reviewerIINAME{}
\def \reviewerIIORCID{}
\def \dateRECEIVED{15 February 2020}
\def \dateACCEPTED{}
\def \datePUBLISHED{}
\def \articleTITLE{[Re] Improved Calibration and Predictive Uncertainty for Deep Neural Networks}
\def \articleTYPE{Replication}
\def \articleDOMAIN{NeurIPS 2019 Reproducibility Challenge}
\def \articleBIBLIOGRAPHY{bibliography.bib}
\def \articleYEAR{2020}
\def \reviewURL{https://openreview.net/forum?id=JhZOkalsiI}
\def \articleABSTRACT{Miscalibration of a model is defined as the mismatch between predicting probability estimates and the true correctness likelihood. In this work, we aim to replicate the results reported by \cite{onmixup} on their analysis of the effect of Mixup \cite{mixup_2018} on a network's calibration. Mixup is an effective yet simple approach of data augmentation, which generates a convex combination of a pair of training images and their corresponding labels as the input and target for training a network. We replicate the results reported by the authors for CIFAR-100 \cite{cifar100}, Fashion-MNIST \cite{fmnist}, STL-10 \cite{stl10}, out-of-distribution and random noise data. Our implementation code can be found at \url{https://github.com/MacroMayhem/OnMixup}}
\def \replicationCITE{S. Thulasidasan, G. Chennupati, J. A. Bilmes, T. Bhattacharya, and S. E. Michalak. “On Mixup Training: Improved Calibration and Predictive Uncertainty for Deep Neural Networks.” In: ArXiv abs/1905.11001 (2019).}
\def \replicationBIB{onmixup}
\def \replicationURL{}
\def \replicationDOI{}
\def \contactNAME{Aditya Singh}
\def \contactEMAIL{aditya.singh@zebra.com}
\def \articleKEYWORDS{rescience c, rescience x, network calibration, mixup, python, pytorch}
\def \journalNAME{ReScience C}
\def \journalVOLUME{6}
\def \journalISSUE{2}
\def \articleNUMBER{10}
\def \articleDOI{10.5281/zenodo.3818605}
\def \authorsFULL{Aditya Singh and Alessandro Bay}
\def \authorsABBRV{A. Singh and A. Bay}
\def \authorsSHORT{Singh and Bay}
\title{\articleTITLE}
\date{}
\author[1]{Aditya Singh}
\author[1,]{Alessandro Bay}
\affil[1]{Zebra Technologies, USA}
