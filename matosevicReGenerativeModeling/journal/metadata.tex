% DO NOT EDIT - automatically generated from metadata.yaml

\def \codeURL{https://github.com/Xemnas0/NCSN-TF2.0}
\def \codeDOI{}
\def \codeSWH{}
\def \dataURL{}
\def \dataDOI{}
\def \editorNAME{Koustuv Sinha}
\def \editorORCID{0000-0002-2803-9236}
\def \reviewerINAME{Anonymous Reviewers}
\def \reviewerIORCID{}
\def \reviewerIINAME{}
\def \reviewerIIORCID{}
\def \dateRECEIVED{15 February 2020}
\def \dateACCEPTED{}
\def \datePUBLISHED{}
\def \articleTITLE{[Re] Generative Modeling by Estimating Gradients of the Data Distribution}
\def \articleTYPE{Replication}
\def \articleDOMAIN{NeurIPS 2019 Reproducibility Challenge}
\def \articleBIBLIOGRAPHY{bibliography.bib}
\def \articleYEAR{2020}
\def \reviewURL{https://openreview.net/forum?id=B1lcYrBgLH}
\def \articleABSTRACT{In this project we attempt to reproduce results from the paper 	extit{Generative Modeling by Estimating Gradients of the Data Distribution} by \citet{ncsn-paper}. The authors propose a novel generative framework based solely on gradients of data density estimated by a neural network. Once the model is trained, sampling can be performed with annealed Langevin dynamics. While we managed to reproduce the experiments qualitatively, we failed to achieve comparable results for Inception and FID scores for CIFAR-10. We further extended the original work in various directions (computing and analysing FID and IS also for CelebA, investigation of the sampling hyperparameters \$\epsilon\$ and \$T\$, linear instead of geometric annealing schedule for noise levels, and different network architecture).}
\def \replicationCITE{Yang Song, Stefano Ermon. "Generative Modeling by Estimating Gradients of the Data Distribution" In 33rd Conference on Neural Information Processing Systems, Vancouver, Canada. (NeurIPS 2019)}
\def \replicationBIB{https://dblp.org/rec/journals/corr/abs-1907-05600.bib}
\def \replicationURL{https://arxiv.org/abs/1907.05600v2}
\def \replicationDOI{}
\def \contactNAME{Antonio Matosevic}
\def \contactEMAIL{matose@kth.se}
\def \articleKEYWORDS{generative model, tensorflow}
\def \journalNAME{ReScience C}
\def \journalVOLUME{6}
\def \journalISSUE{1}
\def \articleNUMBER{}
\def \articleDOI{}
\def \authorsFULL{Antonio Matosevic, Eliisabet Hein and Francesco Nuzzo}
\def \authorsABBRV{A. Matosevic, E. Hein and F. Nuzzo}
\def \authorsSHORT{Matosevic, Hein and Nuzzo}
\title{\articleTITLE}
\date{}
\author[1]{Antonio Matosevic}
\author[1,]{Eliisabet Hein}
\author[1,]{Francesco Nuzzo}
\affil[1]{KTH Royal Institute of Technology, Stockholm, Sweden}
